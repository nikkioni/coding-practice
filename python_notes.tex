\documentclass[10pt, a4paper]{article}
\usepackage{graphicx} % Required if you want to include images
\usepackage{xcolor} % Allows defining custom colours
\usepackage{minted} % Required for syntax highlighting (uses Pygments)

% Set default options for all minted options 
\setminted{
    linenos,             % Show line numbers
    numbersep=8pt,       % Space between numbers and code
    frame=single,        % Put a box/frame around the code
    framesep=2mm,        % Space between frame and code
    fontsize=\footnotesize, % Font size of the code
    breaklines=true,     % Automatically wrap long lines
    tabsize=4,           % How many spaces a tab represents
    style=vs,       % Colour scheme (Visual Studio style here; can change)
}
\usepackage{enumitem}
\setlist[itemize]{leftmargin=1.5cm}
\setlist[enumerate]{leftmargin=1.5cm}

\usepackage{xcolor}
\usepackage{soul} 
\definecolor{hlred}{RGB}{250, 105, 97}

\usepackage{csquotes}

\title{Python Notes}
\author{Nicole Oni}
\date{February 2026}

\begin{document}

\maketitle
\tableofcontents

\newpage

\section{Basics}
\subsection{Introduction}
Python is a popular programming language. It was created by Guido van Rossum, and released in 1991. 

It is used for:
\begin{itemize}
    \item web development (server side), 
    \item software development,
    \item mathematics,
    \item system scripting
\end{itemize}

Python can:
\begin{itemize}
    \item be used on a server to create web applications.
    \item be used alongside software to create workflows.
    \item connect to database systems. It can also read and modify files.
    \item be used to handle big data and perform complex mathematics.
    \item be used for rapid prototyping, or for production-ready software development.
\end{itemize}

Python Syntax compared to other programming languages:
\begin{itemize}
    \item Python was designed for readability, and has some similarities to the English language with influence from mathematics.
    \item Python uses new lines to complete a command, as opposed to other programming languages which often use semicolons or parentheses.
    \item Python relies on indentation, using whitespace, to define scope; such as the scope of loos, functions and classes. Other programming languages often use curly-brackets for this purpose.
\end{itemize}

Other:
\begin{itemize}
    \item The extension for Python files is \mintinline{text}{.py}. 
    \item The command line syntax for checking if Python is installed on your computer (and also to check the python version) is \mintinline{python}{python --version}.
    \item Use \mintinline{python}{exit()} to exit the Python command line interface.
\end{itemize}

\subsection{Python Syntax}
\subsubsection{Python Indentation}
\begin{itemize}
    \item Indentation refers to the spaces at the beginning of a code line.
    \item Where in other programming langugaes the indentation in code is for readability only, the indentation in Python is very important.
    \item Python uses identation to indicate a black of code.
\end{itemize}

For example, Python will give you an error if you skip the indentation:
\begin{minted}[bgcolor=hlred]{python}
if 5 > 2: 
print("Five is greater than two!") 
\end{minted}

Correct version:
\begin{minted}{python}
if 5 > 2: 
    print("Five is greater than two!") 
\end{minted}

\subsubsection{Python Statements}
\begin{itemize}
    \item A \textbf{computer program} is a list of \enquote{instructions} to be \enquote{executed} by a computer.
    \item In a programming language, these programming instructions are called \textbf{statements}. 
    \item In Python, a statement usually ends when the line ends. You do \textit{not} need to use a semicolon (;) like in many other programming languages (for example, Java or C). 
\end{itemize}

Example:
\begin{minted}{python}
print("Python is fun!")
\end{minted}

Many Statements:
\begin{itemize}
    \item Most Python programs contain many statements.
    \item The statements are executed one by one, in the same order as they are written.
\end{itemize}

Example:
\begin{minted}{python}
print("Hello World!")
print("Have a good day.")
print("Learning Python is fun!")
\end{minted}

Semicolons:
\begin{itemize}
    \item Semicolons are optional in Python.
    \item You can write multiple statements on one line by separating them with \mintinline{text}{;} but this is rarely used because it makes it hard to read.
    \item If you put two statements on the same line without a separator (newline or \mintinline{text}{;} , Python will give an error.
\end{itemize}

Example:
\begin{minted}{python}
print("Hello"); print("How are you?"); print("Bye bye!") 
\end{minted}

\setlength{\fboxsep}{4pt} % Controls padding
\colorbox{yellow}{\textbf{Best practice:} Put each statement on its own line so your code is easy to understand.} \\

\subsubsection{Challenge}
Inside the editor, complete the following steps:
\begin{enumerate}
    \item Write a statement that prints \enquote{Hello World!}
    \item Write a statement that prints \enquote{Have a good day.}
    \item Write a statement that prints \enquote{Learning Python is fun!} 
\end{enumerate}

Solution:
\begin{minted}{python}
print("Hello World!")
print("Have a good day.")
print("Learning Python is fun!")
\end{minted}

\subsection{Python Output / Print}
\begin{itemize}
    \item The \mintinline{python}{print()} function can be used to display text or output values.
    \item You can use the \mintinline{python}{print()} function as manytimes as you want. Each cell prints text on a new line by default. 
    \item Text in Python must be inside quotes. You can use either \mintinline{python}{"} double quotes or \mintinline{python}{'} single quotes.
    \item If you forget to put the text inside quotes. Python will give an error. 
\end{itemize}

\begin{minted}[linenos, highlightlines={4}, highlightcolor=hlred]{python}
print("Hello World!")
print("This will work!")
print('This will also work!')
print(This will cause an error)
\end{minted}

\subsubsection{Print Without a New Line}
\begin{itemize}
    \item By default, the \mintinline{python}{print()} function ends with a new line.
    \item If you want to print multiple words on the same line, you can use the \mintinline{python}{end} parameter.
\end{itemize}

Example:
\begin{minted}{python}
print("Hello World!", end=" ")
print("I will print on the same line.")
\end{minted}
Note that we add a space after \mintinline{python}{end=" "} for better readability.

\subsubsection{Print Numbers}
\begin{itemize}
    \item You can also use the \mintinline{python}{print()} function to display numbers.
    \item However, unlike text, we don't put numbers inside double quotes.
    \item You can also do maths inside the \mintinline{python}{print()} function.
\end{itemize}

Example:
\begin{minted}{python}
print(3)
print(358)
print(2*5)
\end{minted}

\subsubsection{Mixing Text and Numbers}
You can combine text and numbers in one output by separating them with a comma. For example:
\begin{minted}{python}
print("I am", 35, "years old.")
\end{minted}

\subsubsection{Challenge}
Inside the editor, complete the following steps:
\begin{enumerate}
    \item Print the text \textbf{\enquote{I am}} and the number \textbf{25} in one print statement.
\end{enumerate}

Solution:
\begin{minted}{python}
print("I am", 25)
\end{minted}


\subsection{Comments}
\begin{itemize}
    \item Python has commenting capability for the purpose of in-code documentation.
    \item Uses of comments:
        \begin{itemize}
            \item Comments can be used to explain Python code.
            \item Comments can be used to make the code more readable.
            \item Comments can be used to prevent execution when testing code. 
        \end{itemize}
    \item Comments start with a \#, and Python will render the rest of the line as a comment.
    \item Comments can be placed at the end of a line, and Python will ignore the rest of the line.
\end{itemize}

Example:
\begin{minted}{python}
# This is a comment
print("Hello, World!") 

print("Hello, World!") # This is a comment 
\end{minted}

\subsubsection{Multiline Comments}
Option 1:
\begin{itemize}
    \item Python does not really have a syntax for multiline comments.
    \item To add a multiline comment you could insert a \# for each line.
\end{itemize}

Option 2:
\begin{itemize}
    \item Or, not quite as intended, you can use a \textbf{multiline string}.
    \item Since Python will ignore string literals that are not assigned to a variable, you can add a multiline string (triple quotes) in your code, and place your comment inside it. 
    \item As long as the string is not assigned to a variable, Python will read the code, but then ignore it, and you have made a multiline comment. 
\end{itemize}

Example: 
\begin{minted}{python}
# This is a comment
# written in 
# more than just one line 

print("Hello, World!") 

"""
This is a comment
written in 
more than just one line 
"""
\end{minted}

\subsubsection{Challenge}
Inside the editor, complete the following steps:
\begin{enumerate}
    \item Add a single-line comment that says \textbf{This is a comment}
    \item Comment out the line \mintinline{python}{print("This should not run")} so it does not execute.
    \item Add a multiline comment (using triple quotes) that says \textbf{This is} \\ \textbf{a multiline} \\ \textbf{comment}
\end{enumerate}

Solution:
\begin{minted}{python}
# This is a comment
# print("This should not run") 
"""
This is 
a multiline 
comment 
"""
\end{minted}

\subsection{Python Variables}
\begin{itemize}
    \item Variables are containers for storing data values.
    \item Python has no command for declaring a variable.
    \item A variable is created the moment you first assign a value to it.
\end{itemize}

Example:
\begin{minted}{python}
x = 5
y = "John"
print(x)
print(y)
\end{minted}

Variables do not need to be declared with any particular \textit{type} and can even change type after they have been set. For example:
\begin{minted}{python}
x = 4 # x is of type int 
x = "Sally" # x is now of type str
print(x)
\end{minted}

\subsubsection{Get the Type}
You can get the data type of a variable with the \mintinline{python}{type()} function. \\
Example:
\begin{minted}{python}
x = 5
y = "John"
print(type(x))
print(type(y))
\end{minted}

\subsubsection{Other}
\begin{itemize}
    \item String variables can be declared either by using single or double quotes.
    \item Variable names are case-sensitive.
\end{itemize}

Example:
\begin{minted}{python}
x = "John" 
# is the same as 
x = 'John' 

a = 4 
A = "Sally"
# A will not overwrite a 
\end{minted}

\subsubsection{Variable Names}
A variable can have a short name (like \mintinline{python}{x} and \mintinline{python}{y}) or a more descriptive name (\mintinline{python}{age, carname, total_volume}. \\
Rules for Python variables:
\begin{itemize}
    \item A variable name must start with a letter or the underscore character.
    \item A variable name cannot start with a number.
    \item A variable name can only contain alpha-numeric characters and underscores (A-z, 0-9, and \_).
    \item Variable names are case-sensitive (\mintinline{python}{age, Age} and \mintinline{python}{AGE} are three different variables). 
    \item A variable name cannot be any of the Python keywords.
\end{itemize}

Legal variable names:
\begin{minted}{python}
myvar = "John"
my_var = "John"
_my_var = "John"
myVar = "John"
MYVAR = "John"
myvar2 = "John"
\end{minted}

Illegal variable names:
\begin{minted}[bgcolor=hlred]{python}
2myvar= "John"
my-var = "John"
my var = "John"
\end{minted}

Multi Words Variable Names: \\
Variable names with more than one word can be difficult to read. There are several techniques you can use to make them more readable:
\begin{itemize}
    \item \textbf{Camel Case:} Each word, except the first, starts with a capital letter: \mintinline{python}{myVariableName = "John"}
    \item \textbf{Pascal Case:} Each word starts with a capital letter: \mintinline{python}{MyVariableName = "John"}
    \item \textbf{Snake Case:} Each word is separated by an underscore character: \mintinline{python}{my_variable_name = "John"}
\end{itemize}

\subsubsection{Assign Multiple Values}
Many Values to Multiple Variables: 
\begin{itemize}
    \item Python allows you to assign values to multiple variables in one line.
    \item Make sure the number of variables matches the number of values, or else you will get an error.
\end{itemize}
Example:
\begin{minted}{python}
x, y, z = "Orange", "Banana", "Cherry" 
print(x)
print(y)
print(z)
\end{minted}

One Value to Multiple Variables: \\
And you can assign the \textit{same} value to multiple variables in one line:
\begin{minted}{python}
x = y = z = "Orange"
print(x)
print(y)
print(z)
\end{minted}

\subsubsection{Unpack a Collection}
If you have a collection of values in a \mintinline{python}{list, tuple} etc. Python allows you to extract the values into variables. This is called \textit{unpacking}. \\

Example (unpack a list):
\begin{minted}{python}
fruits = ["apple", "banana", "cherry"]
x, y, z = fruits
print(x)
print(y)
print(z)
\end{minted}

\subsubsection{Output Variables}
\begin{itemize}
    \item The \mintinline{python}{print()} function is often used to output variables.
    \item In the \mintinline{python}{print()} function, you output multiple variables, separated by a comma.
    \item You can also use the \mintinline{python}{+} operator to output multiple variables.
        \begin{itemize}
            \item For strings, add a space at the end otherwise there will be no spaces between words.
            \item For numbers, the \mintinline{python}{+} character works as a mathematical operator. 
            \item In the \mintinline{python}{print} function, when you try to combine a string and a number with the \mintinline{python}{+} operator, Python will give you and error. 
        \end{itemize}
    \item The best way to output multiple variables in the \mintinline{python}{print()} function is to separate them with commas, which even support different data types.
\end{itemize}

\begin{minted}[linenos, highlightlines={18-21}, highlightcolor=hlred]{python}
# Output multiple variables using a comma 
x = "Python"
y = "is"
z = "awesome"
print(x, y, z)

# Output multiple variables using + 
x = "Python"
y = "is"
z = "awesome"
print(x + y + z)

# Using the + character as a mathematical operator 
x = 5
y = 10
print(x + y)

# Combining strings and numbers with the + operator
x = 5
y = "John"
print(x + y)
\end{minted}

\subsubsection{Global Variables}
\begin{itemize}
    \item Variables that are created outside of a function (as in all the previous examples) are known as \textbf{global variables}. 
    \item Global variables can be used by everyone, both inside of functions and outside.
\end{itemize}

Example: \\
Create a variable outside of a function, and use it inside the function.
\begin{minted}{python}
x = "awesome"

def myfunc():
    print("Python is" + x)

myfunc() 
\end{minted}

If you create a variable with the same name inside a function, this variable will be \textbf{local}, and can only be used inside the function. The global variable with the same name will remain as it was, global and with the original value. \\

Example: \\
Create a variable inside a function, with the same name as the global variable.
\begin{minted}{python}
x = "awesome"

def myfunc():
    x = "fantatic"
    print("Python is " + x)

myfunc()

print("Python is " + x) 
\end{minted}

The global keyword 
\begin{itemize}
    \item Normally, when you create a variable inside a function, that variable is local, and can only be used inside that function.
    \item To create a global variable inside a function, you can use the \mintinline{python}{global} keyword. 
    \item If you use the \mintinline{python}{global} keyword, the variable belongs to the global scope.
\end{itemize}

Example:
\begin{minted}{python}
def myfunc():
    global x 
    x = "fantastic"

myfunc() 

print("Python is " + x)
\end{minted}

To change the value of a global variable inside a function, refer to the variable by using the \mintinline{python}{global} keyword. For example:
\begin{minted}{python}
x = "awesome"

def myfunc():
    global x 
    x = "fantastic"

myfunc() 

print("Python is " + x)
\end{minted}

\subsubsection{Challenge}
Inside the editor, complete the following steps:
\begin{enumerate}
    \item Create a variable \mintinline{python}{x} and assign it the value \textbf{5}.
    \item Create a variable \mintinline{python}{y} and assign it the value \textbf{\enquote{John}}.
    \item Use the \mintinline{python}{type()} function to print the type of \mintinline{python}{x}.
\end{enumerate}

Solution:
\begin{minted}{python}
x = 5
y = "John"
print(type(x))
\end{minted}

\subsection{Python Data Types}
\subsubsection{Built-in Data Types}
\begin{itemize}
    \item In programming, data type is an important concept.
    \item Variables can store data of different types, and different types can do different things.
    \item Python has the following data types built-in by defaul, in these categories:
\end{itemize}

\begin{center}
\begin{tabular}{|l|l|}
    \hline
    Text Type & \mintinline{python}{str} \\
    \hline
    Numeric Types & \mintinline{python}{int, float, complex} \\
    \hline
    Sequence Types & \mintinline{python}{list, tuple, range} \\
    \hline
    Mapping Type & \mintinline{python}{dict} \\
    \hline
    Set Types & \mintinline{python}{set, frozenset} \\
    \hline
    Boolean Type & \mintinline{python}{bool} \\
    \hline
    Binary Types & \mintinline{python}{bytes, bytearray, memoryview} \\
    \hline
    None Type & \mintinline{python}{NoneType} \\
    \hline
\end{tabular}    
\end{center}

\subsubsection{Getting the Data Type}
You can get the data type of any object by using the \mintinline{python}{type()} function.
\begin{minted}{python}
x = 5
print(type(x))
\end{minted}

\begin{center}
\begin{tabular}{|l|l|}
    \hline
    \textbf{Example} & \textbf{Data Type} \\
    \hline
    \mintinline{python}{x = "Hello World"} & \mintinline{python}{str} \\
    \hline
    \mintinline{python}{x = 20} & \mintinline{python}{int} \\
    \hline
    \mintinline{python}{x = 20.5} & \mintinline{python}{float} \\
    \hline
    \mintinline{python}{x = 1j} & \mintinline{python}{complex} \\
    \hline
    \mintinline{python}{x = ["apple", "banana", "cherry"]} & \mintinline{python}{list} \\
    \hline
    \mintinline{python}{x = ("apple", "banana", "cherry")} & \mintinline{python}{tuple} \\
    \hline
    \mintinline{python}{x = range(6)} & \mintinline{python}{range} \\
    \hline
    \mintinline{python}{x = {"name": "John", "age": 36}} & \mintinline{python}{dict} \\
    \hline
    \mintinline{python}{x = {"apple", "banana", "cherry"}} & \mintinline{python}{set} \\
    \hline
    \mintinline{python}{x = frozenset({"apple", "banana", "cherry"})} & \mintinline{python}{frozenset} \\
    \hline
    \mintinline{python}{x = True} & \mintinline{python}{bool} \\
    \hline
    \mintinline{python}{x = b"Hello"} & \mintinline{python}{bytes} \\
    \hline
    \mintinline{python}{x = bytearray(5)} & \mintinline{python}{bytearray} \\
    \hline
    \mintinline{python}{x = memoeryview(bytes(5))} & \mintinline{python}{memoryview} \\
    \hline
    \mintinline{python}{x = None} & \mintinline{python}{NoneType} \\
    \hline
\end{tabular}    
\end{center}

\subsubsection{Setting the Specific Data Type}
If you want to specify the data type, you can use the following constructor functions:
\begin{center}
\begin{tabular}{|l|l|}
    \hline
    \textbf{Example} & \textbf{Data Type} \\
    \hline
    \mintinline{python}{x = str("Hello World")} & \mintinline{python}{str} \\
    \hline
    \mintinline{python}{x = int(20)} & \mintinline{python}{int} \\
    \hline
    \mintinline{python}{x = float(20.5)} & \mintinline{python}{float} \\
    \hline
    \mintinline{python}{x = complex(1j)} & \mintinline{python}{complex} \\
    \hline
    \mintinline{python}{x = list(["apple", "banana", "cherry"])} & \mintinline{python}{list} \\
    \hline
    \mintinline{python}{x = tuple(("apple", "banana", "cherry"))} & \mintinline{python}{tuple} \\
    \hline
    \mintinline{python}{x = range(6)} & \mintinline{python}{range} \\
    \hline
    \mintinline{python}{x = dict(name="John" age=36)} & \mintinline{python}{dict} \\
    \hline
    \mintinline{python}{x = set({"apple", "banana", "cherry"})} & \mintinline{python}{set} \\
    \hline
    \mintinline{python}{x = frozenset({"apple", "banana", "cherry"})} & \mintinline{python}{frozenset} \\
    \hline
    \mintinline{python}{x = bool(5)} & \mintinline{python}{bool} \\
    \hline
    \mintinline{python}{x = bytes(5)} & \mintinline{python}{bytes} \\
    \hline
    \mintinline{python}{x = bytearray(5)} & \mintinline{python}{bytearray} \\
    \hline
    \mintinline{python}{x = memoeryview(bytes(5))} & \mintinline{python}{memoryview} \\
    \hline
\end{tabular}    
\end{center}

\subsubsection{Challenge}
Inside the editor, complete the following steps:
\begin{enumerate}
    \item Create a variable \mintinline{python}{x} with the value \textbf{5}.
    \item Create a variable \mintinline{python}{y} with the value \textbf{3.14}.
    \item Create a variable \mintinline{python}{z} with the value \textbf{\enquote{Hello}}.
    \item Print the data type of each variable using \mintinline{python}{type()}. 
\end{enumerate}

Solution:
\begin{minted}{python}
x = 5
y = 3.14 
z = "Hello" 

print(type(x))
print(type(y))
print(type(z))
\end{minted}

\subsection{Python Numbers}
\begin{itemize}
    \item Int, or integer, is a whole number, positive or negative, without decimals, of unlimited length. 
    \item Float, or \enquote{floating point number} is a number, positive or negative, containing one or more decimals.
    \item Float can also be scientific numbers with an \enquote{e} to indicate the power of 10. 
    \item Complex numbers are written with \enquote{j} as the imaginary part.
    \item You can convert from one type to another with the \mintinline{python}{int()}, \mintinline{python}{float()} and \mintinline{python}{complex()} methods.
\end{itemize}

\begin{minted}{python}
# The three numeric types in Python
x = 1 # int
y = 2.8 # float
y = 35e3 
z = 1j # complex 

# To verify the type of object 
print(type(x))
print(type(y))
print(type(z))

# Convert from one type to another 
a = float(x)
b = int(y)
c = complex(x)
\end{minted}

\hl{You cannot convert complex numbers into another number type.}

\subsubsection{Random Number}
Python does not have a \mintinline{python}{random()} function to make a random number, but Python has a built-in module called \mintinline{python}{random} that can be used to make random numbers. 
\begin{minted}{python}
import random 
print(random.randrange(1,10)) # displays a random number from 1 to 9
\end{minted}

\subsubsection{Challenge}
Inside the editor, complete the following steps:
\begin{enumerate}
    \item Create a variable \mintinline{python}{x} with the value \textbf{5}.
    \item Create a variable \mintinline{python}{y} with the value \textbf{3.14}.
    \item Create a variable \mintinline{python}{z} with the value \textbf{2 + 3j}.
    \item Print the type of each variable using \mintinline{python}{type()}.
\end{enumerate}

Solution:
\begin{minted}{python}
x = 5
y = 3.14 
z = 2 + 3j

print(type(x))
print(type(y))
print(type(z))
\end{minted}

\subsection{Python Casting}
If you want to specify the data type of a variable, this can be done with casting. Python is an object-oriented language, and as such it uses classes to define data types, including its primitive types.\\

Casting in python is therefore done using constructor functions:
\begin{itemize}
    \item \mintinline{python}{int()} - constructs an integer number from:
    \begin{itemize}
        \item an integer literal
        \item a float literal (by removing all decimals)
        \item a string literal (providing the string represents a whole number)
    \end{itemize}
    \item \mintinline{python}{float()} - constructs a float number from:
    \begin{itemize}
        \item an integer literal
        \item a float literal 
        \item a string literal (providing the string represents a float or an integer)
    \end{itemize} 
    \item \mintinline{python}{str()} - constructs a string from a wide variety of data types, including:
    \begin{itemize}
        \item strings
        \item integer literals
        \item float literals
    \end{itemize}
\end{itemize}

Example:
\begin{minted}{python}
x = str(3) # x will be '3'
y = int (3) # y will be 3
z = float(3) # z will be 3.0

print(int(35.88)) # Output is 35
print(float(35)) # Output is 35.0 
print(str(35.82)) # Output is 35.82
\end{minted}

\subsubsection{Challenge}
Inside the editor, complete the following steps:
\begin{enumerate}
    \item Create a variable \mintinline{python}{x} with integer value \textbf{1}.
    \item Convert \mintinline{python}{x} to a float and store it in \mintinline{python}{a}.
    \item Convert \mintinline{python}{x} to a string and store it in \mintinline{python}{b}.
    \item Print \mintinline{python}{a} and \mintinline{python}{b}. 
\end{enumerate}

Solution:
\begin{minted}{python}
# Create an integer 
x = 1

# Convert to float
a = float(x)

# Convert to string 
b = str(x)

# Print values 
print(a)
print(b)
\end{minted}

\subsection{Python Strings}
\begin{itemize}
    \item Strings in python are surrounded by either single quotation marks, or double quotation marks.
    \item \mintinline{python}{'hello'} is the same as \mintinline{python}{"hello"}.
    \item You can display a string literal with the \mintinline{python}{print()} function. 
\end{itemize}

Example:
\begin{minted}{python}
print("Hello")
print('Hello')
\end{minted}

\subsubsection{Quotes Inside Quotes}
You can use quotes inside a string, as long as they don't match the quotes surrounding the string. 

Example:
\begin{minted}{python}
print("It's alright")
print("He is called 'Johnny'")
print('He is called "Johnny"')
\end{minted}

\subsubsection{Assign String to a Variable}
Assigning a string to a variable is done with the variable name followed by an equal sign and the string. \\ 

Example:
\begin{minted}{python}
a = "Hello"
print(a)
\end{minted}

\subsubsection{Multiline Strings}
You can assign a multiline string to a variable by using three single/double quotes. \\

Example:
\begin{minted}{python}
a = """Hello my name is Nicole,
I am 18 years old,
My birthday is June 2nd,
And I am a Gemini."""
print(a)
\end{minted}

\hl{Note: in the result, the line breaks are inserted at the same position as in the code.}

\subsubsection{Strings are Arrays}
\begin{itemize}
    \item Like many other popular programming languages, strings in Python are arrays of unicode characters.
    \item However, Python does not have a character data type, a single character is simply a string with a length of 1.
    \item Square brackets can be used to access elements of the string.
\end{itemize}

Example:
\begin{minted}{python}
# Get the character at position 1 (remember thatthe first character has the position 0)
a = "Hello, World!" 
print(a[1]) # Output would be e 
\end{minted}

\subsubsection{Looping Through a String}
Since strings are arrays, we can loop through the characters in a string, with a \mintinline{python}{for} loop. \\

Example:
\begin{minted}{python}
# Loop through the letters in the word "banana"
for x in "banana":
    print(x)
\end{minted}

\subsubsection{String Length}
To get the length of a string, use the \mintinline{python}{len()} function. \\

Example: 
\begin{minted}{python}
a = "Hello, world!"
print(len(a)) # The len() function returns the length of a string - in this case, 13
\end{minted}

\subsubsection{Check String}
\begin{itemize}
    \item To check if a certain phrase or character is present in a string, we can use the keyword \mintinline{python}{in}. 
    \item To check if a certain phrase or character is NOT present in a string, we can use the keyword \mintinline{python}{not in}. 
\end{itemize}

Example:
\begin{minted}{python}
txt = "The best things in life are free!" 
if "free" in txt:
    print("Yes 'free' is present.")
if "expensive" not in txt:
    print("No, 'expensive' is NOT present.")
\end{minted}

\subsubsection{Slicing Strings}
\begin{itemize}
    \item You can return a rage of characters by using the slice syntax.
    \item Specify the start index and the end index, separated by a colon, to return a part of the string.
\end{itemize}

Example:
\begin{minted}{python}
# Get the characters from position 2 to position 5 (not included)
b = "Hello, World!"
print(b[2:5]) # Output is llo 
\end{minted}

\hl{\textbf{REMEMBER:} The first character has index 0.} \\

Slice From Start: \\
By leaving out the start index, the range will start at the first character. \\


Example:
\begin{minted}{python}
# Get the characters from the start to position 5 (not included)
b = "Hello, World!
print(b[:5]) # Output is Hello
\end{minted}

Slice To the End: \\
By leaving out the end index, the range will go to the end. \\


Example:
\begin{minted}{python}
# Get the characters from position 2 all the way to the end
b = "Hello, World!"
print(b[2:]) # Output is llo, World!
\end{minted}

Negative Indexing: \\
Use negative indexes to start the slice from the end of the string. \\


Example:
\begin{minted}{python}
b = "Hello, World!"
print(b[-5:-2]) # Output is orl (remember does not include -2)
\end{minted}

\subsubsection{String Methods}
Python has a set of built-in methods that you can use on strings:
\begin{itemize}
    \item The \mintinline{python}{upper} method returns the string in upper case. 
    \item The \mintinline{python}{lower} method returns the string in lower case. 
    \item Whitespace is the space before and/or after the actual text, and very often you want to remove this space.
    \item The \mintinline{python}{strip} method removes any whitespace from the beginning or the end. 
    \item The \mintinline{python}{replace} method replaces a string with another string.
    \item The \mintinline{python}{split} method splits the strings into substrings if it finds instances of the separator.
\end{itemize}

Examples:
\begin{minted}{python}
a = "Hello, World!"
print(a.upper()) # Output is HELLO, WORLD!
print(a.lower()) # Output is hello, world!
print(a.replace("H", "J")) # Output is Jello, World!
print(a.split(",")) # Output is ['Hello', 'World!']

b = " Hello, World! " 
print(a.strip()) #Output is "Hello, World!" 
\end{minted}

\subsubsection{String Concatenation}
\begin{itemize}
    \item To concatenate, or combine, two strings you can use the + operator.
    \item To add a space between them, add a \mintinline{python}{" "}
\end{itemize}

Example:
\begin{minted}{python}
a = "Hello"
b = "World"
c = a + b 
d = a + " " + b 
print(c) # Output is HelloWorld
print(d) # Output is Hello World 
\end{minted}

\subsubsection{String Format}
As we learnt in the Python Variables chapter, we \textbf{cannot} combine strings and numbers like this: \\

Example:
\begin{minted}[bgcolor=hlred]{python}
age = 36
txt = "My name is John, I am " + age
print(txt)
\end{minted}

But we can combine strings and numbers by using \textit{f-strings} or the \mintinline{python}{format()} method. \\

F-Strings
\begin{itemize}
    \item F-strings were introduced in Python 3.6, and is now the preferred way of formatting strings.
    \item To specify a string as an f-string, simply put a \mintinline{python}{f} in front of the string literal, and add curly brackets \mintinline{python}{{}} as placeholders for variables and other operations. 
\end{itemize}

Example:
\begin{minted}{python}
age = 36
txt = f"My name is John, I am {age}"
print(txt)
\end{minted}

\section{Classes}
\section{File Handling}
\section{Python Modules}
\section{Python Matplotlib}
\section{Machine Learning}
\section{Python DSA}
\section{Python MySQL}
\section{Python MongoDB}
\section{Python Reference}
\section{Module Reference}
\section{Python How To}
\section{Python Examples}
\end{document}


